\documentclass{article}
\usepackage[utf8]{inputenc}
\usepackage{amsmath}
\usepackage[spanish]{babel}

\setcounter{secnumdepth}{0}

\title{Portal Dirección Informática (La Wiki)}
\author{Elías Rojas}
\date{\today}

\begin{document}

\maketitle

\tableofcontents

\section{Introducción}
El siguiente documento es un relevamiento del portal para la dirección de informática. Se plantean objetivos y estimados del tiempo.

\section{Motivación}
La información dentro de institución está repetida y relativamente desorganizada.

Existen artículos en distintos formatos que no poseen versionado, donde cada efector realiza su propia modificación y 
no queda registro de esto. Los nombres de los mismos no poseen una nomenclatura particular por lo que se da lugar
a realizar varias veces el mismo trabajo con, tal vez, alteraciones pequeñas.

Además, para tener acceso a estos documentos, en algunos casos se depende de una cuenta en particular.

Algunos archivos en formato PDF están escaneados y no pueden ser buscados por contenido.

Por lo tanto, se vislumbra un portal autoritativo, de rápido acceso y organizado para palear las problemáticas nombradas.

\section{Objetivos}
Los objetivos del portal serán los siguientes:
\begin{itemize}
    \item Desarrollar un portal común, seguro y accesible desde cualquier lugar, de manera ágil. Preferentemente solo con usuario y contraseña.
    \item Distribución de permisos por usuario y por grupos (perfiles). 
    \item Poseer un visualizador de documentos, para agilizar la visualización de los mismos.
    \item Poseer búsqueda por contenido de artículos en PDFs.
    \item Poseer búsqueda por secciones o artículos dentro del portal.
    \item En etapas avanzadas, acceso al NAS correspondiente para disponer de un abanico de documentos más grande.
\end{itemize}

\section{Herramientas}
Debido a la personalización requerida por la aplicación, la estructura general monolítica de los servicios 
dentro de la municipalidad, en contraste a, por ejemplo, una arquitectura de microservicios,
la relativa sencillez y la abundancia de documentación (por el tiempo que llevan siendo utilizadas), 
se optará por utilizar las siguientes herramientas de código abierto:

Todas las herramientas están sujetas a cambios.
\begin{itemize}
    \item \textbf{Django}: Framework con modelo Vista-Controlador. Este generará el esqueleto de las vistas y mantendrá la lógica de los servicios.
    \item \textbf{HTMX}: Librería de frontend que permite utilizar AJAX directamente en HTML. Pudiendo definir una página utilizando únicamente HTML y CSS sin la necesidad de utilizar javascript.
    \item \textbf{MariaDB}: Base de datos.
    \item \textbf{PdfImage}: Librería encargada de convertir archivos con formato PDF a formatos de imágenes JPEG, PNG,...
    \item \textbf{Pytesseract}: Librería de Python utilizada para el reconocimiento óptico de caracteres (OCR, en inglés) que trabaja sobre formatos de imágen. Se utilizará en combinación con PdfImage para tener la posibilidad de parsear pdfs escaneados, pasandolos primero a una imágen.
    \item \textbf{Django all-auth}: Libreria de Django que resuelve la gestion de autenticacion e integracion de servicios con terceros.
\end{itemize}

\section{Tiempos}
Debido a que el equipo no posee experiencia con desarrollo utilizando Django, se utilizarán dos semanas para tomar familiaridad con la herramienta.
La familiaridad con las librerías se irá tomando durante el transcurso del desarrollo. 

El tiempo estimado para tener un producto viable con estas caracteristicas será alrededor de \textbf{2 meses}.
Este tiempo incluye los periodos de familiarización y se considera una buena oportunidad para implementar herramientas
fieles que podrán servir durante un periodo considerable.

Las entregas de avances en la aplicación se realizarán con los jefes de área en la dirección respectiva.
Se pretende una participación activa una vez el desarrollo esté relativamente avanzado respecto a los objetivos planteados en esta documentación.

\section{Etapas}
Las etapas del desarrollo estarán dispuestas de la siguiente forma:
\begin{enumerate}
    \item Diagramado de base de datos.
    \item Autenticación, autorización y manejo de usuarios.
    \item Secciones y artículos. Estos deberían poder ser agregados por un administrador o el usuario con permisos correspondientes.
    \item Carga y previsualización de archivos.
    \item Búsqueda por secciones y nombres de artículos.
\end{enumerate}

\end{document}
